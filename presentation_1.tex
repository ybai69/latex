%%%%%%%%%%%%%%%%%%%%%%%%%%%%%%%%%%%%%%%%%
% Beamer Presentation
% LaTeX Template
% Version 1.0 (10/11/12)
%
% This template has been downloaded from:
% http://www.LaTeXTemplates.com
%
% License:
% CC BY-NC-SA 3.0 (http://creativecommons.org/licenses/by-nc-sa/3.0/)
%
%%%%%%%%%%%%%%%%%%%%%%%%%%%%%%%%%%%%%%%%%

%----------------------------------------------------------------------------------------
%	PACKAGES AND THEMES
%----------------------------------------------------------------------------------------

\documentclass{beamer}

\mode<presentation> {

% The Beamer class comes with a number of default slide themes
% which change the colors and layouts of slides. Below this is a list
% of all the themes, uncomment each in turn to see what they look like.

%\usetheme{default}
%\usetheme{AnnArbor}
%\usetheme{Antibes}
%\usetheme{Bergen}
%\usetheme{Berkeley}
%\usetheme{Berlin}
%\usetheme{Boadilla}
%\usetheme{CambridgeUS}
%\usetheme{Copenhagen}
%\usetheme{Darmstadt}
%\usetheme{Dresden}
%\usetheme{Frankfurt}
%\usetheme{Goettingen}
%\usetheme{Hannover}
%\usetheme{Ilmenau}
%\usetheme{JuanLesPins}
%\usetheme{Luebeck}
\usetheme{Madrid}
%\usetheme{Malmoe}
%\usetheme{Marburg}
%\usetheme{Montpellier}
%\usetheme{PaloAlto}
%\usetheme{Pittsburgh}
%\usetheme{Rochester}
%\usetheme{Singapore}
%\usetheme{Szeged}
%\usetheme{Warsaw}

% As well as themes, the Beamer class has a number of color themes
% for any slide theme. Uncomment each of these in turn to see how it
% changes the colors of your current slide theme.

%\usecolortheme{albatross}
%\usecolortheme{beaver}
%\usecolortheme{beetle}
%\usecolortheme{crane}
%\usecolortheme{dolphin}
%\usecolortheme{dove}
%\usecolortheme{fly}
%\usecolortheme{lily}
%\usecolortheme{orchid}
%\usecolortheme{rose}
%\usecolortheme{seagull}
%\usecolortheme{seahorse}
%\usecolortheme{whale}
%\usecolortheme{wolverine}

%\setbeamertemplate{footline} % To remove the footer line in all slides uncomment this line
%\setbeamertemplate{footline}[page number] % To replace the footer line in all slides with a simple slide count uncomment this line

%\setbeamertemplate{navigation symbols}{} % To remove the navigation symbols from the bottom of all slides uncomment this line
}

\usepackage{graphicx} % Allows including images
\usepackage{booktabs} % Allows the use of \toprule, \midrule and \bottomrule in tables

%----------------------------------------------------------------------------------------
%	TITLE PAGE
%----------------------------------------------------------------------------------------


%Slide 1: name of R package; a very brief description; 
 %        R package authors; your names.

%  Please center these and use a large enough font to be visible at the 
%  back of the room. 

%In slides 2-4, address the following points:

%1. Briefly describe the data science challenge.
%2. Briefly describe (some of) the functionality 
%3. Show an instance of a calculation, e.g. through a graphical display




\title[RCurl]{RCurl} % The short title appears at the bottom of every slide, the full title is only on the title page

%\author{Duncan Temple Lang} % Your name

\institute[] % Your institution as it will appear on the bottom of every slide, may be shorthand to save space
{
\begin{block}{}
\large{The RCurl package is an R-interface to the libcurl library that provides HTTP facilities. This allows us to download files from Web servers, post forms, use HTTPS (the secure HTTP), use persistent connections, upload files, use binary content, handle redirects, password authentication, etc.}
\end{block}

%\medskip

\large{Author: Duncan Temple Lang}
\medskip

\large{Victoria Mansfield (Speaker) , Yue Bai,  }\\ % Your institution for the title page
%\medskip
%\textit{john@smith.com
%} % Your email address
}
\date{} % Date, can be changed to a custom date

\begin{document}

\begin{frame}
\titlepage % Print the title page as the first slide


\end{frame}

%\begin{frame}
%\frametitle{Data Science Challenge} % Table of contents slide, comment this block out to remove it
%\tableofcontents % Throughout your presentation, if you choose to use \section{} and \subsection{} commands, these will automatically be printed on this slide as an overview of your presentation
%\end{frame}

%----------------------------------------------------------------------------------------
%	PRESENTATION SLIDES
%----------------------------------------------------------------------------------------

%------------------------------------------------
%\section{First Section} % Sections can be created in order to organize your presentation into discrete blocks, all sections and subsections are automatically printed in the table of contents as an overview of the talk
%%------------------------------------------------
%
%\subsection{Subsection Example} % A subsection can be created just before a set of slides with a common theme to further break down your presentation into chunks

%\begin{frame}
%\frametitle{Paragraphs of Text}
%Sed iaculis dapibus gravida. Morbi sed tortor erat, nec interdum arcu. Sed id lorem lectus. Quisque viverra augue id sem ornare non aliquam nibh tristique. Aenean in ligula nisl. Nulla sed tellus ipsum. Donec vestibulum ligula non lorem vulputate fermentum accumsan neque mollis.\\~\\
%
%Sed diam enim, sagittis nec condimentum sit amet, ullamcorper sit amet libero. Aliquam vel dui orci, a porta odio. Nullam id suscipit ipsum. Aenean lobortis commodo sem, ut commodo leo gravida vitae. Pellentesque vehicula ante iaculis arcu pretium rutrum eget sit amet purus. Integer ornare nulla quis neque ultrices lobortis. Vestibulum ultrices tincidunt libero, quis commodo erat ullamcorper id.
%\end{frame}

%------------------------------------------------


%
%------------------------------------------------

\begin{frame}
\frametitle{Data Science Challenge}

\begin{block}{Web Scraping}
\begin{itemize}
\item Getting a page's HTML code and parsing through to get data that we want.
\item  Posting forms and request to the Internet.
\end{itemize}
\end{block}

\begin{block}{R-language interface}
Given the increasing role of HTTP and Web connectivity, and the desire to use R in ways that can access data and services in other domains via HTTP, a more general, flexible and complete R-language interface to client-side HTTP is desirable. The RCurl package provides such an interface for R. 
\end{block}



%\begin{block}{Block 3}
%Suspendisse tincidunt sagittis gravida. Curabitur condimentum, enim sed venenatis rutrum, ipsum neque consectetur orci, sed blandit justo nisi ac lacus.
%\end{block}
\end{frame}

%------------------------------------------------

\begin{frame}
\frametitle{Basic Functionality}
There are three high-level functions in RCurl: getURL(), getForm(), and postForm().

\begin{itemize}
\item  getURL() and getURI(): These functions download one or more URIs (a.k.a. URLs). Input the URL and return the HTML of the Web page.
\medskip
\item getForm() and postForm(): These functions provide facilities for submitting an HTML form. Used for automating retrieval of data sets that might otherwise require a form submission for each data set.
%\item Getting URIs: The most common use of HTTP is to download static or ?fixed content? files. The function getURL() or getURI() provides a simple mechanism to do this.
%\item Forms: There are two functions in RCurl that can be used to submit HTML forms via HTTP: getForm() and postForm()
%\item Options controlling the request: 
%\item Callback Options: There are several callback options in RCurl corresponding to different events in libcurl. These are writefunction, headerfunction, debugfunction, progressfunction.
\end{itemize}
\end{frame}

%
%\begin{frame}
%\frametitle{Multiple Columns}
%\end{frame}
%
%\begin{frame}
%\frametitle{Functionality}
%%The RCurl package provides an interface to the libcurl facilities. libcurl is, as the name suggests, a library programmed in C that provides portable tools for accessing URIs via HTTP requests and other protocols.
%There are three high-level functions in RCurl: getURL(), getForm(), and postForm().
%\begin{block}{Quantifying the Value of Data}
%Most of the data we generate is not directly usable. To make it usable we first need to identify it, then curate it and finally make it accessible. Historically this work was done because data was actively collected and the collection of data was such a burden in itself that its curation was less of an additional overhead. We require data-desalination before it can be consumed.
%\end{block}
%
%\begin{block}{Block 2}
%Pellentesque sed tellus purus. Class aptent taciti sociosqu ad litora torquent per conubia nostra, per inceptos himenaeos. Vestibulum quis magna at risus dictum tempor eu vitae velit.
%\end{block}
%
%%\begin{block}{Block 3}
%%Suspendisse tincidunt sagittis gravida. Curabitur condimentum, enim sed venenatis rutrum, ipsum neque consectetur orci, sed blandit justo nisi ac lacus.
%%\end{block}
%\end{frame}


%------------------------------------------------
%\section{Second Section}
%%------------------------------------------------
%
%\begin{frame}
%\frametitle{Table}
%\begin{table}
%\begin{tabular}{l l l}
%\toprule
%\textbf{Treatments} & \textbf{Response 1} & \textbf{Response 2}\\
%\midrule
%Treatment 1 & 0.0003262 & 0.562 \\
%Treatment 2 & 0.0015681 & 0.910 \\
%Treatment 3 & 0.0009271 & 0.296 \\
%\bottomrule
%\end{tabular}
%\caption{Table caption}
%\end{table}
%\end{frame}
%
%%------------------------------------------------
%
%\begin{frame}
%\frametitle{Theorem}
%\begin{theorem}[Mass--energy equivalence]
%$E = mc^2$
%\end{theorem}
%\end{frame}

%------------------------------------------------

\begin{frame}[fragile] % Need to use the fragile option when verbatim is used in the slide
%\vspace{-2em}

\frametitle{Example}
\begin{example}[getURL(). ]
\footnotesize{
\begin{verbatim}
# Three tables (total area, land area, and water area) in this Web pages.
# We want to know the land area of the 50 United States. 
x<-getURL("https://simple.wikipedia.org/wiki/List_of_U.S._states_by_area")
\end{verbatim}
\vspace{-2.5em}
\begin{figure}
\includegraphics[width=1\linewidth]{222}
\end{figure}

\vspace{-2em}

\begin{columns}[c] % The "c" option specifies centered vertical alignment while the "t" option is used for top vertical alignment

\column{.5\textwidth} % Left column and width
\begin{verbatim}
(area<-readHTMLTable(x))
\end{verbatim}
\vspace{-2em}
\begin{figure}
\includegraphics[width=0.95\linewidth]{333}
\end{figure}



\column{.5\textwidth} % Right column and width
\begin{verbatim}
(land_area<-area[[2]])
\end{verbatim}
\vspace{-2em}
\begin{figure}
\includegraphics[width=0.95\linewidth]{444}
\end{figure}

\end{columns}

%\begin{figure}
%\includegraphics[width=0.4\linewidth]{111}
%\end{figure}
\end{example}


%\begin{example}[getForm(), postForm().]
%\begin{verbatim}
%url.exists("http://www.wisc.edu")
%# [1] TRUE
%\end{verbatim}
%\end{example}
\end{frame}

%------------------------------------------------

%\begin{frame}
%\frametitle{Figure}
%Uncomment the code on this slide to include your own image from the same directory as the template .TeX file.
%\end{frame}

%------------------------------------------------

%\begin{frame}[fragile] % Need to use the fragile option when verbatim is used in the slide
%\frametitle{Citation}
%An example of the \verb|\cite| command to cite within the presentation:\\~
%
%This statement requires citation \cite{p1}.
%\end{frame}

%------------------------------------------------

%\begin{frame}
%\frametitle{References}
%\footnotesize{
%\begin{thebibliography}{99} % Beamer does not support BibTeX so references must be inserted manually as below
%\bibitem[Smith, 2012]{p1} John Smith (2012)
%\newblock Title of the publication
%\newblock \emph{Journal Name} 12(3), 45 -- 678.
%\end{thebibliography}
%}
%\end{frame}

%------------------------------------------------

%\begin{frame}
%\Huge{\centerline{The End}}
%\end{frame}

%----------------------------------------------------------------------------------------

\end{document} 